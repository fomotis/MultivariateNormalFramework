\documentclass[a4paper,12pt]{article}

\renewcommand{\baselinestretch}{1.1}
\setlength{\parindent}{0cm}

%%% Add packages here
\usepackage{times}
\usepackage[utf8]{inputenc}
\usepackage{graphics}
\usepackage{graphicx}
\usepackage{lscape}
\usepackage{amsfonts}
\usepackage{amsmath}
\usepackage{amsthm}
\usepackage{array}
\usepackage{amssymb}
\usepackage{latexsym}
\usepackage{verbatim}
\usepackage{color}
\usepackage{xcolor}
\usepackage{fancyhdr}
\usepackage{fancybox}
\usepackage{mathtools}
\usepackage{graphics}
\usepackage[colorlinks,citecolor=red,linkcolor=black]{hyperref}
\usepackage{subfig}
%\usepackage{w-thm}

\usepackage[super,comma,sort&compress]{natbib}
\bibliographystyle{apalike}
\usepackage{float}
\usepackage[utf8]{inputenc}
\usepackage[english]{babel}
\usepackage{multicol}
%\usepackage[style=numeric]{biblatex}

%%%%%%%%%%%%%%%%%%%%%%%%%%%%%%%%%%%%%

%%% Margins
%\setlength{\bibsep}{2pt}
%\setlength{\bibhang}{2em}

\addtolength{\oddsidemargin}{-.50in}
\addtolength{\evensidemargin}{-.50in}
\addtolength{\textwidth}{1.0in}
\addtolength{\topmargin}{-.40in}
\addtolength{\textheight}{0.80in}

%%% Header
\pagestyle{fancy}
%\chead{\groupname}
\rhead{Traits and Abundance}
\lhead{Letter}
\cfoot{\thepage}
\renewcommand{\headrulewidth}{1.9pt}

\begin{titlepage}
	\title{
		\begin{flushleft} 
			\Huge{A framework to evaluate the effects of environmental change on trait diversity (inter and intra)
			and population abundance.} \\ 
		\vspace{0.4in} \small{Oluwafemi D. Olusoji$^{1,2}$, Thomas Neyens$^{1}$, Marc Aerts$^{1}$, Frederik De Laender$^{2}$}
		\end{flushleft}
	}
	\date{}
\end{titlepage}

\begin{document}
	\maketitle
	\newpage
	\begin{abstract}
			
		\noindent \textit{Keywords: }
	\end{abstract}
	
	
\section*{INTRODUCTION}

\section*{THE FRAMEWORK}

\subsection*{SINGLE SPECIES CASE}

\begin{enumerate}
	\item A trait $T$ of dimension $n \times m$ is measured across different environmental gradient repeatedly over time. Thus, at every time point we have a $n \times m$ matrix of trait values.
	\item An abundance $N$ is also measured repeatedly over time. At every time point, we have a single abundance value.
	\item The trait is assumed normally distributed, i.e. $T_t \sim N(\mu_t, \sigma^2_{T})$ at every time point $t$.
	\item Following from $3$, $\bar{T}$
\end{enumerate}

\subsection*{TWO SPECIES CASE}

\section*{RESULTS}

\section*{DISCUSSION}

\section*{ACKNOWLEDGEMENTS}

\section*{AUTHORSHIP}

\section*{REFERENCES}
	\bibliography{references}
	
\end{document}